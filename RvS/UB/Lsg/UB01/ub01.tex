\documentclass{article}
\author{Max Springenberg, 177792}
\title{
    RvS UB01\\
    Gruppe 4
}
\date{}
\usepackage{amsmath}
\usepackage{amssymb}
\usepackage{stmaryrd}
\usepackage{graphicx}
\setcounter{section}{1}
% \Theta \Omega \omega
\newcommand{\tab}{\null \qquad}
\newcommand{\lA}{$\leftarrow$}
\newcommand{\ue}{$\infty$}
\newcommand{\gap}{\\ \ \\}

\begin{document}
\maketitle
\newpage

\subsection{Quizfragen}
\subsubsection{
    Der Rechnerbeauftragte installiert auf allen Computern des Universitätsnetzes 
    eine einheitliche Version der Office-Software. Ergibt sich daraus ein verteiltes System?
}
Wie ist ein verteiltes System aufgebaut?\\
\\
\begin{tabular}{|lp{0.6\linewidth}|}
    \hline
    Rechnernetz: & 
        Autonome Rechnerknoten, die durch Telekommunikationssysteme verbunden sind.\\
    \hline
    Telekommunikationssystem: &
        System, das seinen Teilnehmern Kommunikationsdienste anbietet.\\
    \hline
    Verteiltes System: &
        Anwendung, mit Komponenten, die an verschiedenen Orten sind.
        Die Komponenten sind im Rechnerknoten installiert.
        Ausfuehrung der Komponenten von Rechnerknoten aus.
        Kommunikation mithilfe eine Telekommunikationssystems.\\
    \hline
\end{tabular}\\
\\
Fuer ein verteiltes System fehlt der Aspekt der Telekommunikation. Es werden Komponenten auf
einzelnen Rechnerknoten installiert, dennoch ist es nicht der Fall, dass beim Ausfuehren auf
einen anderen Rechnerknoten zugegriffen/ oder mit einem anderem Rechnerknoten kommuniziert wird.\\
\subsubsection{
    Ist die Post ein verbindungsloser oder ein vermittlungsorietierter Dienst?
}
\subsubsection{
    Welche Fehlerklassen können bei der Kommunikation per Post auftreten, 
    unter der Annahme, dass das Briefgeheimnis gewahrt wird? Welche nicht?
}
\subsubsection{
    Welche Probleme können auftreten, wenn man in einem verteilten System den globalen Zustand des
    Gesamtsystems bestimmen will?
}
\subsection{Zeitablaufsdiagramm}
\subsection{Simplex vs Duplex v Halbduplex}

\end{document}
