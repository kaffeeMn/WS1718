\documentclass{article}
\author{Max Springenberg, 177792}
\title{
    RvS Skript\\
    Notizen
}
\date{}
\usepackage{amsmath}
\usepackage{amssymb}
\usepackage{stmaryrd}
\usepackage{graphicx}
% \Theta \Omega \omega
\newcommand{\tab}{\null \qquad}
\newcommand{\lA}{$\leftarrow$}
\newcommand{\ue}{$\infty$}
\newcommand{\gap}{\\ \ \\}

\begin{document}
\maketitle
\newpage

\section{Einfuehrung, Internet und Protokolle}
\subsection{Systeme}
\subsubsection{Wie ist ein verteiltes System aufgebaut?}
\begin{tabular}{|lp{0.6\linewidth}|}
    \hline
    Rechnernetz: & 
        Autonome Rechnerknoten, die durch Telekommunikationssysteme verbunden sind.\\
    \hline
    Telekommunikationssystem: &
        System, das seinen Teilnehmern Kommunikationsdienste anbietet.\\
    \hline
    Verteiltes System: &
        Anwendung, mit Komponenten, die an verschiedenen Orten sind.
        Die Komponenten sind im Rechnerknoten installiert.
        Ausfuehrung der Komponenten von Rechnerknoten aus.
        Kommunikation mithilfe eine Telekommunikationssystems.\\
    \hline
\end{tabular}\\
\subsubsection{Nachteile verteilter Systeme}
\begin{tabular}{|lp{0.6\linewidth}|}
    \hline
    Kopplung &
        Kommunikation findet selten statt, dementsprechend die Synchronosationsrate schwach
        und eine Fehlertoleranz mkuss existieren.\\
    \hline
    Nebenlaeufigkeit/ Concurrency &
        Es existieren weitestgehend unabhaengige Fortschritte.\\
    \hline
    Dezentrale Kontrolle &
        Autonomie der Rechnerknoten, lolkale Kontrolle auf der Basis partieller Sichten.
        Eine Vollstaendige Sicht auf das globale System wird vermieden.\\
    \hline
\end{tabular}\\
\subsection{Kommunikationsdienste}
\subsubsection{Kommunikationsformen}
\begin{tabular}{|lp{0.6\linewidth}|}
    \hline
    Unicast &
        Kommunikation mit einem Partner.\\
    \hline
    Multicast &
        Kommunikation mit mehreren Partnern (Gruppe).\\
    \hline
    Broadcast &
        Kommunikation mit ALLEN.\\
    \hline
\end{tabular}\\
\subsubsection{Rihtungsbetrieb}
\begin{tabular}{|lp{0.6\linewidth}|}
    \hline
    Simplex & 
        Ein Simplex sendet Nachrichten von Sender nach Empfaenger.\\
    \hline
    Duplex & 
        Ein Duplex sendet Nachrichten an einen Empfaenger und empfaengt auch Nachrichten.
        Dies passiert vorallem auch simultan.\\
    \hline
    Halbduplex & 
        Ein Halbduplex sendet und empfaengt Nachrichten wie auch ein Duplex, jedoch passiert
        dies nie simultan.\\
    \hline
\end{tabular}
\subsubsection{Dienstleistende Systeme}
Bei dienstleistenden Systemen gilt es zu differenzieren zwischen Dienstnehmern und Dienstbringern.\\
Kommunikation zwischen Dienstnehmern wickelt Dienstleistungen ab.\\
Der Diensterbringer interpretiert Nachrichten nicht.\\
\\
Wichtige Eigenschaften von Dienstleistungen sind:\\
\tab Partneradressierung\\
\tab Datagramme (UDP)\\
\tab Verbindungsorientierung (TCP)\\
\tab Unicast / Multicast / Broadcast\\
\tab Simplex / Duplex / Halbduplex
\subsubsection{Sichten}
statisch:\\
\tab Darstellung/ Gliederung in Dienstzugangspunkte\\
\\
dynamisch:\\
\tab Darstellung/ Gliederung in Dienstzugangspunkte\\
\tab zusaetzlicher Aspekt des Zeitlichen Ablaufs.\\
\tab \tab Dienststimuli und Dienstreaktion gehoeren zur Sicht.\\
\subsubsection{Nachrichtenreihenfolge}

\end{document}
