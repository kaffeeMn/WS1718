\documentclass{article}
\author{Max Springenberg, 177792}
\title{
    Wahrscheinlichkeitsrechnung Und Mathematische Statistik\\
    WS 1718
}
\date{}
\usepackage{amsmath}
\usepackage{amssymb}
\usepackage{stmaryrd}
\usepackage{graphicx}
% \Theta \Omega \omega
\newcommand{\tab}{\null \qquad}
\newcommand{\lA}{$\leftarrow$}
\newcommand{\ue}{$\infty$}
\newcommand{\gap}{\\ \ \\}
\newcommand{\img}[1]{\includegraphics[\width=1/\textwidth]{#1}}
\newcommand{\headline}[1]{\textbf{\textit{#1}}\\}

\begin{document}
\maketitle
\section{Merkmale und Datentypen}
Datentypen\\
\begin{tabular}{ll}
    Skalentyp   &Aussagen\\
    \hline
    Nominal     &Gleich/Verschieden\\
    Ordinal     &Groesser/Kleiner\\
    \hline
    Interval    &Differenz\\
    Verhaeltnis &Verhaeltnis\\
\end{tabular}\\
\gap
Diskrete Datentypen sind endlich oder abzaehlbar unendlich\\
Stetige Datentypen sind ueberabzaehlbar viele\\
\gap
Generell koennen Datentypen unter Informationsverlust in Datentypen niederer
Ordnung ueberfuehrt werden,\\
\section{Tabellerische und grafische Darstellung von univarianten Daten}
\subsection{Quantitativ diskrete Daten}
\begin{tabular}{ll}
    $M_N =\{e_1, ..., e_N\}$    &Population bestehend aus Objekten $e_i$\\
    $X$                         &Quantitatives Merkmal\\
    $x,x \in W_x$               &Merkmalsauspraegung von X\\
    $W_x = \{x(1),...,x(J)\}$   &Wertebereich von X mit Merkmalsauspraegung\\
    $D_N = \{x_1,...,x_N\}$     &Urliste aus der Messung von X in der Population $M_N$\\
\end{tabular}
\subsubsection{Formeln}
\headline{Absolute Haeufigkeit}
$
N_j$ von $x(j):\\
N_j = N[x(j)] = \sum_{i=1}^J d_i(j), I_{x(e_i)=x(j)}\\
\Rightarrow N = \sum_{j=1}^J N_j\\
$
Hierbei is $d_i$ die i-te Spalte und $x(j)$ das j-te Object, dem Werte zugeordnet werden.\\
\gap
\headline{Relative Haeufigkeit}
$
f_j$ von $x(j) = \frac{N_j}{N}\\
\Rightarrow \sum_{j=1}{J} f_j= 1\\
$
\gap
\headline{Empirische Verteilungsfunktion}
$
F_N(x) = \begin{cases}
    0 & , x < x(1)\\
    s_j = \sum_{k=1}^j f_k & , x(1) \geq x\\
\end{cases}
$
Diese Funktion steigt wie eine Treppe und nimmt den Wert bis zum naechst kleineren x.\\
\subsection{Quantitativ stetige Daten}
\begin{tabular}{ll}
    $M_N =\{e_1, ..., e_N\}$                &Population bestehend aus Objekten $e_i$\\
    $X$                                     &Quantitatives Merkmal\\
    $x,x \in W_x$                           &Merkmalsauspraegung von X\\
    $W_X = (-\inf,\inf) = \bigcup_{j=1}^J K_j$ &Klassierter (vorallem kategorisierter) Wertebereich von X\\
    $K_j = (v_{j-1},V_j), K_J=(v_{J-1},v{J})$ &Merkmalsklassen mit Klassengrenzen\\
    $D_N = \{x_1,...,x_N\}$                 &Urliste aus der Messung von X in der Population $M_N$\\
\end{tabular}
\subsection{Histogramm}

\end{document}
