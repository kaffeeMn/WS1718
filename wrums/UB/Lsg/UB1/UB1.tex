\documentclass{article}
\author{Max Springenberg, 177792}
\title{
    WRUMS UB01
}
\date{}
\usepackage{amsmath}
\usepackage{amssymb}
\usepackage{stmaryrd}
\usepackage{graphicx}
\setcounter{section}{1}
% \Theta \Omega \omega
\newcommand{\tab}{\null \qquad}
\newcommand{\lA}{$\leftarrow$}
\newcommand{\ue}{$\infty$}
\newcommand{\gap}{\\ \ \\}

\begin{document}
\maketitle
\newpage

\subsection{Merkmale und Datentypen}
a)\\
\\
Die Skalennieveaus ergeben sich wie folgt\\
\\
\begin{tabular}{ll}
    Herstellungsdatum&      Intervall\\
    Prozessor&              Nominal\\
    Prozessornummer&        Ordinal\\
    Cachegroeße&            Verhaeltniss\\
    Bildschirmdiagonale&    Verhaeltniss\\
    Aufloesung&             Verhaeltniss\\
\end{tabular}\\
\gap
b)\\
Die Skalennieveaus ergeben sich wie folgt\\
\\
\begin{tabular}{ll}
    Name&           Nominal\\
    Geburtsdatum&   Intervall\\
    Alter&          Verhaeltniss\\
    Geschlecht&     Nominal\\
    Augenfarbe&     Nominal\\
    Blutgruppe&     Nominal\\
    Gewicht&        Verhaeltniss\\
    Groesse&        Verhaeltniss\\
    Bew. d. Ges. Zust.& Intervall\\
    Koerpertemperatur& Intervall\\
\end{tabular}\\
\end{document}
