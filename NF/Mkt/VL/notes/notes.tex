\documentclass{article}
\author{Max Springenberg, 177792}
\title{
    NEbenfach Markt und Absatz\\
    Marketing WS 1718
}
\date{}
\usepackage{amsmath}
\usepackage{amssymb}
\usepackage{stmaryrd}
\usepackage{graphicx}
% \Theta \Omega \omega
\newcommand{\tab}{\null \qquad}
\newcommand{\lA}{$\leftarrow$}
\newcommand{\ue}{$\infty$}
\newcommand{\gap}{\\ \ \\}
\newcommand{\img}[1]{\includegraphics[\width=1/\textwidth]{#1}}
\newcommand{\headline}[1]{\textbf{\textit{#1}}\\}

\begin{document}
\maketitle
\section{Grundsaetzliches zur Betriebswirtschaftslehre}
Grundlegende Bausteine der BWL sind\\
\begin{enumerate}
    \item MarketingPlanung
    \item Produktpolitik
    \item Vertriebspolitik
    \item Kommunikationsplitik
    \item Konsumentenverhalten
    \item Marketingforschung
    \item Sektorales Marketing
\end{enumerate}
\subsection{Definitionen und Modelle}
Def: \headline{Wirtschaft}
\\
Als Betrieb bezeichnet man eine planvoll organisierte Wirtschaftseinheit.
in der Produktionsfaktoren kombiniert werden, um Gueter unde Dienstleistungen herzustellen und 
abzusetzen.\\
\gap
Mod: \headline{Geschaeftsleitung}
\\
Man betrachte die Geschaeftslöeitung als Menge von Submodellen/ Instanzen.\\
$
Ge      := $Geschaeftsleitung$\\
Be      := $Beschaffung$\\
Pr      := $Produktion$\\
Mkt     := $Marketing$\\
Psnl    := $Personal$\\
Fin     := $Finanzen$\\
\gap
Ge = \{
    Be, Pr, Mkt, Psnl, Fin, R
\}
$$mit R als restliche Instanzen$$\\
R = Ge \setminus \{
    Be, Pr, Mkt, Psnl, Fin
\}\\
|R| \geq 0\\
$
\end{document}
