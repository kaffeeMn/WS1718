\documentclass{article}
\author{Max Springenberg, 177792}
\title{
    Marketing Uebung\\
    01 ABC-Analyse
}
\date{}
\usepackage{amsmath}
\usepackage{amssymb}
\usepackage{stmaryrd}
\usepackage{graphicx}
\usepackage{caption}

% \Theta \Omega \omega
\newcommand{\tab}{\null \qquad}
\newcommand{\lA}{$\leftarrow$}
\newcommand{\ue}{$\infty$}
\newcommand{\gap}{\\ \ \\}

\setcounter{section}{1}

\begin{document}
\maketitle
\newpage

\subsection{Sie als Mitglied einer studentischen Unternehmensberatung haben als
Datengrundlage die durchschnittlichen monatlichen Umsätze für die Bearbeitung
dieser Aufgabe erhalten. Betrachten Sie das gegenwärtige Leistungsprogramm und
unterteilen Sie die Produktgruppen anhand von Umsatzanteilen in A-, B- und C-
Bereiche. Ziehen Sie die Grenze bitte bei etwa 70 \% des Gesamtumsatzes für den A-
Bereich und bei etwa 95 \% für den B-Bereich. Berechnen Sie bitte die Prozentsätze
mit zwei Dezimalstellen.}

\subsection{}

\subsection{Nach der Berechnung ist es Ihre Aufgabe die Ergebnisse vor dem Marketing und
Vertriebsmanagement zu präsentieren. Stellen Sie Ihre Ergebnisse graphisch dar.
Kennzeichnen Sie alle relevanten Stellen.}

\subsection{Der Direktor für Marketing und Vertrieb möchte nun auch eine ABC-Analyse der
Kunden durchführen. Nachdem A-, B- und C-Kunden identifiziert wurden, ist dem
Verantwortlichen jedoch unklar, wie diese Kundengruppen differenziert
behandelt werden können. Empfehlen Sie dem Direktor, wie man bei den A- und
B-Kunden vorgehen könnte. Begründen Sie Ihren Ratschlag kurz.}
\end{document}
